% Inspired in a file by Winston Chang 2012

\documentclass[10pt,landscape]{article}
\usepackage{multicol}
\usepackage{calc}
\usepackage{ifthen}
\usepackage[landscape]{geometry}



% This sets page margins to .5 inch if using letter paper, and to 1cm
% if using A4 paper. (This probably isn't strictly necessary.)
% If using another size paper, use default 1cm margins.
\ifthenelse{\lengthtest { \paperwidth = 11in}}
	{ \geometry{top=.5in,left=.5in,right=.5in,bottom=.5in} }
	{\ifthenelse{ \lengthtest{ \paperwidth = 297mm}}
		{\geometry{top=1cm,left=1cm,right=1cm,bottom=1cm} }
		{\geometry{top=1cm,left=1cm,right=1cm,bottom=1cm} }
	}

% Turn off header and footer
\pagestyle{empty}
 

% Redefine section commands to use less space
\makeatletter
\renewcommand{\section}{\@startsection{section}{1}{0mm}%
                                {-1ex plus -.5ex minus -.2ex}%
                                {0.5ex plus .2ex}%x
                                {\normalfont\large\bfseries}}
\renewcommand{\subsection}{\@startsection{subsection}{2}{0mm}%
                                {-1explus -.5ex minus -.2ex}%
                                {0.5ex plus .2ex}%
                                {\normalfont\normalsize\bfseries}}
\renewcommand{\subsubsection}{\@startsection{subsubsection}{3}{0mm}%
                                {-1ex plus -.5ex minus -.2ex}%
                                {1ex plus .2ex}%
                                {\normalfont\small\bfseries}}
\makeatother

% Define BibTeX command
\def\BibTeX{{\rm B\kern-.05em{\sc i\kern-.025em b}\kern-.08em
    T\kern-.1667em\lower.7ex\hbox{E}\kern-.125emX}}

% Don't print section numbers
\setcounter{secnumdepth}{0}


\setlength{\parindent}{0pt}
\setlength{\parskip}{0pt plus 0.5ex}


% -----------------------------------------------------------------------

\begin{document}

\raggedright
\footnotesize

\begin{center}
     \Large{\textbf{LCG Quick Reference}} \\
\end{center}

\section{Protocols}
\begin{multicols}{2}
% multicol parameters
\setlength{\columnseprule}{0.25pt}
\setlength{\premulticols}{1pt}
\setlength{\postmulticols}{1pt}
\setlength{\multicolsep}{1pt}
\setlength{\columnsep}{2pt}

\subsubsection{Action potential protocol}
{\tt lcg-ap -a <amplitude>} \\
Injects a single (brief) pulse of current. Its purpose is to elicit a single spike to extract its salient electrophysiological properties. The duration of the pulse should be short enough in order not to interfere with the actual shape of the action potential. \\

\begin{tabular}{ll}
\hline\hline
\multicolumn{1}{c}{\textbf{Switch}} & \multicolumn{1}{c}{\textbf{Usage}}\\ 
\hline
{\bf-a} & Amplitude of the stimulation (in pA)\\
-d & Duration of the stimulation (default 1 ms)\\
-n & Number of repetitions (default 20)\\
-i & Interval between repetitions (default 1 sec)\\
-F & Sampling frequency\\
-I & Input channel\\
-O & Output channel\\
--rt & Use real-time system (yes or no)\\
\end{tabular}

\subsubsection{V-I curve protocol}
{\tt lcg-vi -a -200,50,50 -d 1 --without-preamble} \\
Presents the cell with hyperpolarizing pulses between the specified values. The purpuse is to characterize the cells passive properties by measuring the Voltage-Current relation. 

\begin{tabular}{ll}
\hline\hline
\multicolumn{1}{c}{\textbf{Switch}} & \multicolumn{1}{c}{\textbf{Usage}}\\ 
\hline
-a & stimulation amplitudes (default -300,50,50 pA)\\
-d & stimulation duration (default 3 s)\\
-I & input channel\\
-O & output channel\\
-F & sampling frequency\\
--rt & use real-time system (yes or no, default yes)\\
--without-preamble & do not include stability preamble (see manual)\\
--no-shuffle & do not shuffle trials\\
--no-kernel & do not compute the electrode kernel (AEC)\\
\end{tabular}

\subsubsection{Time constant protocol}
Delivers hyperpolarizing pulses of current and can be used to compute the time constant of a cell.

\begin{tabular}{ll}
\hline\hline
\multicolumn{1}{c}{\textbf{Switch}} & \multicolumn{1}{c}{\textbf{Usage}}\\ 
\hline
-a & Amplitude of the stimulation (default -300 pA)\\
-d & Duration of the stimulation (default 10 ms)\\
-n & Number of reperitions (Default 30)\\
-I & input channel\\
-O & output channel\\
-F & sampling frequency\\
--rt & use real-time system (yes or no, default yes)\\
--no-kernel & do not compute the electrode kernel (AEC)\\
\end{tabular}

\subsubsection{Current ramp protocol}

\subsubsection{White noise protocol}

\subsubsection{Filtered noise protocol}

\subsubsection{Input resistance}

\subsubsection{DC steps protocol}

\subsubsection{f-I curve protocol} 

\subsubsection{Frequency clamp protocol}

\subsubsection{Spontaneous activity protocol}

\subsubsection{Train of pulses protocol}

\subsubsection{Utility programs}

\end{multicols}

\subsection{Entities}
Table with all entities name, description, parameters, input, output
\subsection{Streams}
Table with all streams name, description, parameters, input, output

\section{Waveforms}
Waveforms use the STIM notation which consists of rows of 12 numbers in a text file.

DC, constant value, waveform\\
\begin{tabular}{rrrrrrrrrrrr}
[T & 1 & Amplitude & -- & -- & -- & -- & -- & -- & -- & -- & EXPON]
\end{tabular}

Ornstein-Uhlenbeck stochastic process realization\\

\begin{tabular}{rrrrrrrrrrrr}
[T & 2 & Mean  & Standard deviation & Time constant & - & - & FIXSEED & MYSEED & - & - & EXPON]
\end{tabular}

Sinusoidal waveform\\

\begin{tabular}{rrrrrrrrrrrr}
[T & 3 & Amplitude & Frequency (Hz) & Phase (rad) & - & - & FIXSEED (0/1) & SEED & - & - & EXPON]
\end{tabular}


Square wave, with zero mean\\

\begin{tabular}{rrrrrrrrrrrr}
[T & 4 & Absolute maximum & Frequency (Hz) & Duty-cycle (percent of the period) & - & - & - & - & - & - & EXPON]
\end{tabular}


Saw tooth wave, with zero mean\\

\begin{tabular}{rrrrrrrrrrrr}
[T & 5 & Absolute maximum & Frequency (Hz) & Duty-cycle (percent of the period) & - & - & - & - & - & - & EXPON]
\end{tabular}


Sine waveform with frequency sweep\\

\begin{tabular}{rrrrrrrrrrrr}
[T & 6 & Amplitude & Starting frequency (Hz) & Ending frequency (Hz) & - & - & - & - & - & - & EXPON]
\end{tabular}


Ramp waveform, increasing or decreasing slope\\

\begin{tabular}{rrrrrrrrrrrr}
[T & 7 & Final amplitude & - & - & - & - & - & - & - & - & EXPON]
\end{tabular}


Poisson / regular unipolar square pulses\\

\begin{tabular}{rrrrrrrrrrrr}
[T & 8 & Amplitude & Frequency (Hz) & Pulse width (ms) & - & - & FIXSEED (0/1) & SEED & - & - & EXPON]
\end{tabular}


Poisson / regular unipolar exponentially decaying pulses\\

\begin{tabular}{rrrrrrrrrrrr}
[T & 9 & Amplitude & Frequency (Hz) & Pulse width (ms) & - & - & FIXSEED (0/1) & SEED & - & - & EXPON]
\end{tabular}


Poisson / regular bipolar square pulses (zero mean)\\

\begin{tabular}{rrrrrrrrrrrr}
[T & 10 & Amplitude & Frequency (Hz) & Pulse width (ms) & - & - & FIXSEED (0/1) & SEED & - & - & EXPON]
\end{tabular}


Uniformly distributed stochastic process realisation\\

\begin{tabular}{rrrrrrrrrrrr}
[T & 11 & Mean & Standard deviation & - & - & - & FIXSEED (0/1) & SEED & - & - & EXPON]
\end{tabular}


Double exponential decay (Alpha waveform)\\

\begin{tabular}{rrrrrrrrrrrr}
[T & 12 & Amplitude & Rise time constant (ms) & Decay time constant (ms) & - & - & FIXSEED (0/1) & SEED & - & - & EXPON]
\end{tabular}


\rule{0.3\linewidth}{0.25pt}
\scriptsize
http://danielelinaro.github.io/dynclamp/
Daniele Linaro, Jo\~ao Couto and Michele Giugliano
\end{document}
