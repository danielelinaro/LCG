This chapter describes some additional commands and features present
in \progname that can be quite useful in a number of situations.

\section{lcg-seal-test} \label{sec:seal_test}

\texttt{lcg-seal-test} is a resistance test to aid establishing/monitoring the seal in patch-clamp experiments. It can be extremely useful when the amplifier does not provide a graphical interface to monitor the resistance trough the pipette. Both voltage clamp and current clamp modes are available, depending on the selected mode a voltage/current step is delivered through the pipette and the resistance is calculated and constantly updated on screen. 
This program supports the following options:

\begin{itemize}
\item{\texttt{-a}} Amplitude of the pulses 
\item{\texttt{-d}} Duration of the pulses
\item{\texttt{-i}} Inter-pulse interval (ms)
\item{\texttt{--CC}} Current clamp mode.
\item{\texttt{--H}} Hold value; the resistance test pulse if injected on top of this value.
\item{\texttt{--reset}} runs lcg-zero before quitting, this option can be useful when we want to control the injected 
\item{\texttt{--kernel}} Computes the kernel for AEC in current clamp mode only
\item{\texttt{--remote-blind-patch server:port,axes}} blind patch clamp interface for a remote TCP-IP controller.
\item{\texttt{--patch-opts [hunt depth, max depth, step size, hunt step size]}}  Controls the options for blind patch clamp.
\end{itemize}

\paragraph{GUI control} When the program is started it opens a graphical user interface with the following controls:
\begin{itemize}
\item{\emph{Reset}} brings the GUI to the initial trace.
\item{\emph{Auto Zoom}} Always update the plot to display the entire traces
\item{\emph{Fit Decay}} fits a double exponential to the current trace.
\item{\emph{Running}} pauses the resistance test.
\item{\emph{Save}} to MAT file, saves the resistance test pulses and fit information to a file.
\end{itemize}

You can use the \textbf{+} and \textbf{-} keys to increase/decrease the holding potential/current by 
10mV (VC) or 25 pA (CC). Note that if the \texttt{--reset} option was set, this value is reset.

\paragraph{Blind patch clamping} This feature is experimental, contact Jo\~ao Couto if you want to use them.


\section{lcg-experiment-launcher} \label{sec:exp_launcher}
The \texttt{lcg-experiment-launcher} is a dynamical graphical user interface that allows protocol standardization and automates some commands that would otherwise need to be re-typed every experiment. It also provides a way to append metadata to experiments in a reproducible way, or to manage projects that require a wide range of different protocols. \\
In addition it provides a versatile way to abstract from the \emph{command line interface} and to quickly plot experiment files. \\
It can be evoked using the command \texttt{lcg-experiment-launcher} and is composed of three tabs: 
\begin{itemize}
\item{\emph{General parameters}} Allows creating an experiment folder following a standardized nomenclature specified by the experimenter (left panel) and appending custom \emph{metadata} to that experiment (right panel).
	
\item{\emph{Protocols}} Allows running experiment commands a.k.a. \emph{protocols} and organizes the results in folders.

\item{\emph{Data display}} Allows visualization of the resulting data (similarly to the command \texttt{lcg-plot}).
\end{itemize}


\section{Experiment folder} \label{sec:exp_folder}

\section{Metadata storage} \label{sec:metadata}
As mentioned previously, the actual command that performs the
low-level work of parsing configuration files, instantiating and
connecting entities and running the recording/simulation is called
\verb+lcg-experiment+. Among other things, this program is also
responsible for saving all the metadata required to
unambiguously reconstruct, a posteriori, what kind of protocol was
performed. In order to achieve this goal, for each H5 file it saves,
\verb+lcg-experiment+ also creates a corresponding subdirectory,
contained in a hidden directory named \verb+.lcg+ and located in the
same folder as the H5 file. For example, if you saved a file called
\verb+my_recording.h5+ in the folder
\verb+/home/john+, then \verb+lcg-experiment+ will
first create the folder \verb+/home/john/.lcg+ and
then the additional subfolder
\verb+/home/john/.lcg/my_recording+. \\
This folder contains the following files:
\begin{itemize}
\item The XML configuration file used by \verb+lcg-experiment+.
\item All the stim-files referenced in the XML configuration file.
\item A script called \verb+replay+ that can be used to run
  \verb+lcg-experiment+ with exactly the same command-line arguments
  it was called when performing the experiment. This allows keeping
  track, among other things, of the number of repetitions and of the
  inter-trial interval.
\item A file called \verb+hashes.sha+ that contains the SHA-1 checksum
  of the H5 file, of the configuration file, of the stim-files and of
  the \verb+replay+ script. This makes it possible to verify whether
  such files have been modified (for instance, a parameter in the
  configuration file might have been changed) after the experiment was
  performed.
\end{itemize}
The integrity of the files can be verified with the program
\verb+sha1sum+. For example, assume that you have recorded
$10\,\second$ of spontaneous activity from one channel with the
command
\begin{lstlisting}
lcg-stimulus -l 10 -I 0 -o spontaneous.h5
\end{lstlisting}
The command
\begin{lstlisting}
ls -alh
\end{lstlisting}
will produce (among a lot of other files) the following output:
\begin{verbatim}
drwxr-xr-x  3 daniele users 4.0K Jan  7 15:35 .lcg
-rw-r--r--  1 daniele users 1.4K Jan  7 15:35 stimulus.xml
-rw-r--r--  1 daniele users 466K Jan  7 15:35 spontaneous.h5
\end{verbatim}
If we now move into the \verb+.lcg+ directory and list its contents
with the commands
\begin{lstlisting}
cd .lcg
ls -lh
\end{lstlisting}
we get the following output:
\begin{verbatim}
drwxr-xr-x 2 daniele users 4.0K Jan  7 15:35 spontaneous
\end{verbatim}
We can now move into the \verb+spontaneous+ directory, list its
contents and use \verb+sha1sum+ in the following way:
\begin{lstlisting}
cd spontaneous
ls -lh
sha1sum -c hashes.sha
\end{lstlisting}
which will produce the following output:
\begin{verbatim}
-r--r--r-- 1 daniele users  218 Jan  7 15:35 hashes.sha
-rwxr-xr-x 1 daniele users   53 Jan  7 15:35 replay
-rw-r--r-- 1 daniele users 1.4K Jan  7 15:35 stimulus.xml
-rw-r--r-- 1 daniele users   32 Jan  7 15:35 tmp.stim
\end{verbatim}
and
\begin{verbatim}
../../spontaneous.h5: OK
stimulus.xml: OK
tmp.stim: OK
replay: OK
\end{verbatim}
which informs us that no file has been changed since it was
written to disk.

\section{Adding comments to data files}
\progname provides a command, called \verb+lcg-annotate+, that can be
used to add comments to existing H5 files. The syntax is the
following:
\begin{verbatim}
lcg-annotate [<options> ...] file
\end{verbatim}
where file is an H5 file generated by \progname. \verb+lcg-annotate+
accepts the \verb+-m+ option (or the long version \verb+--message+)
which allows passing the comment on the command line. If the comment
is composed of several words, these should be enclosed in quotes.

If no comment or file are specified, \verb+lcg-annotate+ will prompt
the user for the necessary information.
