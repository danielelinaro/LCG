This manual describes the functionalities of \progname, a suite of
programs -- called {\it commands} -- that can be used to
perform electrophysiological experiments.\\
\progname was developed by Daniele Linaro and Jo\~ao Couto while at
the Theoretical Neurobiology and Neuroengineering Laboratory of the
University of Antwerp. Michele Giugliano developed the concept and
meta-description of stimulus files and implemented the first version
of the related source code.

The main features of \progname are the following:
\begin{itemize}
\item dynamic clamp capabilities, with built-in active electrode
compensation \cite{Brette:2008};
\item straightforward design and implementation of closed loop and
hybrid experiments;
\item command-line operability;
\item ease of automation by scripting;
\item compact description of stimulation waveforms;
\item simple implementation of non real time protocols (e.g., current
  and voltage clamp);
\item support for multiple real time engines;
\item simple installation and operation procedures.
\end{itemize}

This manual is organized as follows: Chap.~\ref{chap:installation}
covers in detail the installation of a real-time operating system and of
\progname and its dependencies. Chapter~\ref{chap:start} explains how
to start using \progname. Chapter~\ref{chap:protocols} describes
the most commonly used commands that come with \progname and the
related electrophysiological protocols, while Chap.~\ref{chap:stimgen} explains how
stimulus files can be used to apply arbitrary stimulation protocols,
in traditional voltage and current clamp
experiments. Chapter~\ref{chap:datafiles} covers the internal
organization of the data files saved by \progname and contains a few
examples that show how to load \progname data files using both Matlab
and Python. Chapters~\ref{chap:configuration} through
\ref{chap:entities} explain how configuration files can be used to
describe custom experimental protocols, and which basic building blocks
-- called {\it entities} or {\it streams} -- are available in \progname.
Finally, Chap.~\ref{chap:features} covers some additional features of
\progname.

In the simplest scenario, in which \progname is already installed on a
machine and the experimentalist only wants to perform conventional
voltage and/or current clamp experiments, without the need for real
time control over the experiment, Chapters~\ref{chap:protocols} and
\ref{chap:stimgen} with some parts of Chap.~\ref{chap:datafiles} are
sufficient to get going.