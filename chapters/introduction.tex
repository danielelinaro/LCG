This manual describes the functionalities of \progname, a suite of
programs -- called {\it commands} -- that can be used to
perform electrophysiological experiments.\\
\progname was developed by Daniele Linaro and Jo\~ao Couto while at
the Theoretical Neurobiology and Neuroengineering Laboratory of the
University of Antwerp. Michele Giugliano developed the concept and
meta-description of stimulus files and implemented the first version
of the related source code.

The main features of \progname are the following:
\begin{itemize}
\item dynamic clamp capabilities, with built-in active electrode
compensation \cite{Brette:2008};
\item straightforward design and implementation of closed loop and
hybrid experiments;
\item command-line operability;
\item ease of automation by scripting;
\item compact description of stimulation waveforms;
\item simple implementation of non real time protocols (e.g., current
  and voltage clamp);
\item support for multiple real time engines;
\item simple installation and operation procedures.
\end{itemize}

\section{What is LCG?}
%
%\paragraph{}
\progname is a software toolbox that can be used to perform
electrophysiological experiments using simple stimulation protocols
but also more complex paradigms such as dynamic clamp and/or hybrid networks.
\progname is made up of two parts: the first one is a C++ library that implements the
low-level objects responsible, among other things, of analog
input/output, saving data to disk and interfacing with the real-time
scheduling system of the OS. The second part is a Python interface
that handles simpler tasks, mostly related to the high-level
description of specific electrophysiological protocols and simple data
analysis and plotting.

% If you are an experienced programmer and you want to interface with lib\progname\, consult the Doxygen documentation and the source code in \href{https://github.com/danielelinaro/dynclamp}{GitHub}.

%%%%%%%%%%%%%%%%%%%%%%%%%%%%%%%%%%%%%%%%%%%
% The following is commented out
%\iffalse

\progname consists of a set of {\it commands}, each of which performs
a specific task.
The entry point of all \progname commands is the program called
\verb+lcg+. Its purpose is merely to parse its arguments and
call the appropriate protocol with the arguments with which it was
invoked. To clarify this concept, let's look at the following example:
\begin{lstlisting}
lcg steps -a -200,100,50 -d 2
\end{lstlisting}
This command instructs \verb+lcg+ to look for a program called
\verb+lcg-steps+ in the directories where executable files are
located,\footnote{These are usually stored in the \inlineCode{\$PATH}
environment variable.} and invoke it with the subsequent arguments
unchanged. The previous call is therefore equivalent to
\begin{lstlisting}
lcg-steps -a -200,100,50 -d 2
\end{lstlisting}
This approach is particularly suited to adding extensions to
\progname: to do so, it is sufficient to follow the naming scheme just
described and place additional scripts or programs where
\verb+lcg+ can find them, i.e., anywhere in the path.
 
\progname comes with a \verb+help+ command, that can be used to
display general information about \progname and a list of the most
commonly used protocols, by simply invoking
\begin{lstlisting}
lcg help
\end{lstlisting}
Alternatively, \verb+lcg-help+ accepts one single argument,
which must be the name of a recognized command and invokes it with
the \verb+-h+ option. For example, the following four commands
produce the same result:
\begin{lstlisting}
lcg help steps
lcg-help steps
lcg steps -h
lcg-steps -h
\end{lstlisting}
For this to work, all commands must accept the \verb+-h+ option
and interpret it as a request for help. All commands that come with
\progname adhere to this convention, and additional ones should
do the same.

This manual is organized as follows: Chap.~\ref{chap:installation}
covers in detail the installation of a real-time operating system and of
\progname and its dependencies. Chapter~\ref{chap:start} explains how
to start using \progname. Chapter~\ref{chap:protocols} describes
the most commonly used commands that come with \progname and the
related electrophysiological protocols, while Chap.~\ref{chap:stimgen} explains how
stimulus files can be used to apply arbitrary stimulation protocols,
in traditional voltage and current clamp
experiments. Chapter~\ref{chap:datafiles} covers the internal
organization of the data files saved by \progname and contains a few
examples that show how to load \progname data files using both Matlab
and Python. Chapters~\ref{chap:configuration} through
\ref{chap:entities} explain how configuration files can be used to
describe custom experimental protocols, and which basic building blocks
-- called {\it entities} or {\it streams} -- are available in \progname.
Finally, Chap.~\ref{chap:features} covers some additional features of
\progname.

In the simplest scenario, in which \progname is already installed on a
machine and the experimentalist only wants to perform conventional
voltage and/or current clamp experiments, without the need for real
time control over the experiment, Chapters~\ref{chap:start}, \ref{chap:protocols} and
\ref{chap:stimgen} with some parts of Chap.~\ref{chap:datafiles} are
sufficient to get going.