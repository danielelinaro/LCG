In this chapter, we describe the electrophysiology protocols that come
with \progname.

All protocols allow the user to
specify the sampling frequency (usually with a \verb+-F+ switch,
but be careful of exceptions) and the input and output channels
(usually with the \verb+-I+ and \verb+-O+ options, respectively,
but be \emph{extremely} careful of exceptions).

Of the following protocols, only those for the computation of f-I
curves with a Proportional-Integral-Derivative (PID) controller and
for clamping the frequency of a neuron require a real-time kernel. For
all other protocols, a Comedi installation is sufficient.

Note that, whenever in the following voltage clamp or current clamp
are mentioned, \progname does not communicate with the amplifier to
switch modes: it is up to the user to do so.

An up-to-date list of the more commonly used protocols can be obtained
by entering
\begin{lstlisting}
lcg help
\end{lstlisting}

\section{Action potential protocol}
This protocol consists in injecting a brief (the default value is
$1\,\milli\second$) depolarizing pulse of current to elicit a single
action potential to extract its salient electrophysiological
properties. The duration of the pulse should be short enough in order
not to interfere with the actual shape of the action potential. The
user can specify the amplitude and duration of the pulse, as well as
the number of repetitions and the interval between them.\\
For example, the following command
\begin{lstlisting}
lcg ap -a 2000 -d 1 -n 10 -i 5
\end{lstlisting}
injects $10$ pulses of amplitude $2000\,\pico\ampere$ and duration
$1\,\milli\second$ with intervals of $5\,\second$. The voltage is
recorded for $0.5\,\second$ before and after the application of the pulse.

\section{V-I curve protocol}
This protocol consists in injecting a series of hyperpolarizing and/or
depolarizing steps of current to compute a V-I curve. The default
is to inject steps of current from $-300$ to $50\,\pico\ampere$ in
steps of $50\,\pico\ampere$, which is suited for cortical pyramidal
neurons or other large types of cells. Alternatively, the user can
specify the values of current by using the \verb+-a+ option.\\
For example, the following command
\begin{lstlisting}
lcg vi -a -100,40,20 -d 1
\end{lstlisting}
instructs the protocol to inject steps of current from $-100$ to
$40\,\pico\ampere$ in steps of $20\,\pico\ampere$ with a duration of
$1\,\second$.

\section{Time constant protocol}
This protocol is identical to the action potential protocol described
previously, with the sole exception that the injected pulses of
current are hyperpolarizing and can therefore be used to compute the
time constant of a cell. The default values of amplitude and duration
of the pulses are $-300\,\pico\ampere$ and $10\,\milli\second$, but
they can be changed using the \verb+-a+ and \verb+-d+
options, respectively. \\
For example, the command
\begin{lstlisting}
lcg tau -n 20
\end{lstlisting}
instructs the protocol to inject 20 pulses, instead of the default 30.

\section{Current ramp protocol}
This protocol injects an increasing ramp of current into a cell, to
find its threshold for spiking. The user can specify the initial and
final values of the ramp, as well as its duration.\\
For example, the command
\begin{lstlisting}
lcg ramp -a 50 -A 300 -d 15
\end{lstlisting}
instructs the protocol to inject a $15\,\second$ ramp, starting from
$50\,\pico\ampere$ and ending at $300\,\pico\ampere$.

\section{White noise protocol}
This protocol injects white noise into a cell. It is primarily used to
compute the electrode kernel for the Active Electrode Compensation
technique described in \cite{Brette:2008}.\\
For example, the command
\begin{lstlisting}
lcg kernel -s 100
\end{lstlisting}
instructs the protocol to inject a noisy trace lasting $10\,\second$,
with zero mean and standard deviation of $100\,\pico\ampere$. The
default value of standard deviation is $250\,\pico\ampere$, which is
suited for cortical pyramidal cells. The script also calls
\verb+lcg-extract-kernels+ to extract the electrode kernel and
saves the result in a file called \verb+kernel.dat+. An
additional option is provided (\verb+-H+) to inject an
additional \emph{holding} current that is not taken into account in
the actual computation of the electrode kernel. This option is useful
for spontaneously active cells, like Purkinje cells, or for \emph{in
vivo} experiments.

\section{Filtered noise protocol}
This protocol injects a Ornstein-Uhlenbeck process into a cell. It can
be used either to analyze the reliability and precision of cells, as
was done in \cite{Mainen:1995}, or to extract the parameters of an
exponential integrate and fire model, as described in
\cite{Badel:2008}.\\
For example, the command
\begin{lstlisting}
lcg ou -n 20 -i 5 -m 100 -s 100 -t 20 -d 3
\end{lstlisting}
instructs the protocol to inject 20 repetitions of a noisy current
with mean and standard deviation equal to $100\,\pico\ampere$ and time
constant of $20\,\milli\second$. The duration of each injection is
$3\,\second$ and the interval between repetitions is $5\,\second$.

\section{Input resistance}
This protocol injects a train of hyperpolarizing pulses suited to
compute the input resistance of a cell \emph{in vivo}, as described in
\cite{Crochet:2006}.\\
For example, the command
\begin{lstlisting}
lcg rin -a -100 -d 300 -f 1 -D 90
\end{lstlisting}
instructs the protocol to inject $300\,\milli\second$-long pulses of
amplitude $-100\,\pico\ampere$ at a frequency of $1\,\hertz$. The
total duration of the recording is specified by the \verb+-D+
option, in this case $90\,\second$. Alternatively, the user can
specify, with the \verb+-N+ option, the number of pulses to
inject.

\section{DC steps protocol}
This protocol injects DC steps of current into a cell, and is a
generalization of the previously described protocol for the
computation of the V-I curve. It has additional parameters that allow
to specify the duration of the recording after the application of the
stimulus and an optional holding value.\\
For example, the command
\begin{lstlisting}
lcg steps -a 100,400,50 -d 2 -n 4 -i 5
\end{lstlisting}
instructs the protocol to inject steps of current from $100$ to
$400\,\pico\ampere$ in steps of $50\,\pico\ampere$ with a duration of
$2\,\second$ and interval of $5\,\second$. Each amplitude is repeated
4 times. Additionally, this protocol has a \verb+--vclamp+
option, which instructs the program to consider the steps as voltage ones
and use, as a consequence, the appropriate default conversion factors
(contained in the \verb+.lcg-env.sh+ file, see
Sec.~\ref{sec:configuration}). In this case, the command
\begin{lstlisting}
lcg steps -a -100,0,20 --vclamp
\end{lstlisting}
instructs the program to inject \emph{voltage} steps at values ranging from
$-100$ to $0\,\milli\volt$, in steps of $20\,\milli\volt$. For all the
other options, default values are used.\\
Note that in both cases, the amplitudes of the steps are shuffled, in
order to minimize the effects due to adaptation to increasing (or
decreasing) stimulation amplitudes. However, an option is provided
(\verb+--no-shuffle+) to inject the steps in a linear order.

\section{f-I curve protocol}\label{sec:fi}
Input-output curves, also known as f-I curves, can be easily computed
with \progname in two ways: the most straightforward and traditional
one is to use \verb+lcg-steps+ to inject steps of current and
subsequently count the number of spikes emitted for each amplitude.

An alternative way is to use a PID
controller and have the neuron follow an increasing ramp of
frequency. The protocol that implements this algorithm is
\verb+lcg-fi+, which requires a real-time kernel to operate correctly
and can be called as follows:
\begin{lstlisting}
lcg fi -a 150 -m 5 -M 30 -T 30 -n 2 -w 60
\end{lstlisting}
which instructs the protocol to initially inject a current of
$150\,\pico\ampere$ and drive the cell from a minimum of $5$ to a
maximum of $30\,\hertz$. The protocol is repeated twice, with each
trial lasting $30\,\second$ and an interval of $60\,\second$ between
trials. Note that here the inter-trial interval option is specified by
the \verb+-w+ option, instead of the more common \verb+-i+, which is
used, together with \verb+-p+ and \verb+-d+, to specify the integral,
proportional and derivative gains of the controller,
respectively. Note also that, for optimal results, the initial value
of current ($150\,\pico\ampere$ in the example) should be chosen such
that it elicits spiking in the cell at a frequency as close as
possible to the minimum required value (here equal to $5\,\hertz$).

\section{Frequency clamp protocol}\label{sec:fclamp}
In a sense, this protocol is the dual of an injection of
suprathreshold current: in that case, a constant current is injected
(clamped) and the firing frequency of the cell can be measured. Here,
the protocol finds the current necessary to make the cell spike at a
given frequency. To do so, it uses a PID controller in much the same
way as the protocol for the computation of a f-I curve, with the
difference that here the value of frequency is a constant instead of a
ramp.\\
For example, the command
\begin{lstlisting}
lcg fclamp -f 10 -a 100
\end{lstlisting}
instructs the protocol to clamp the firing frequency of the cell at a
value of $10\,\hertz$, starting with an injected current of
$100\,\pico\ampere$, which must be above threshold, in order for the
frequency estimator to work properly.

\section{Spontaneous activity protocol}
This protocol can be used to record spontaneous activity from an
arbitrary number of input channels and is suited both for \emph{in
vitro} and \emph{in vivo} experiments. The user can specify the
duration of the recording, the device, subdevice, channels and gains
to be used in the recording. The only limitation is that all channels
need be on the same device and subdevice.\\
For example, the command
\begin{lstlisting}
lcg spontaneous -d 60 -I 0,1,2,3
\end{lstlisting}
instructs the protocol to record for $60\,\second$ from channels 0, 1,
2 and 3. If not specified, the default device file and input subdevice
are taken from the \verb+.lcg-env.sh+ file, see
Sec.~\ref{sec:configuration}.

\section{Train  of pulses protocol}
This protocol injects a train of brief depolarizing protocols into a
cell to elicit action potentials, while simultaneously recording from
two cells, the stimulated one and another that might be connected to
it. The purpose of this protocol is therefore to test the connectivity
between pairs of cells and to measure their short-term synaptic
properties.\\
For example, the command
\begin{lstlisting}
lcg pulses -N 8 -O 0 -I 0,1 -f 30 -d 1 -A 3000
\end{lstlisting}
instructs the protocol to stimulate a presynaptic cell with a train of
8 pulses of duration $1\,\milli\second$ and amplitude
$3000\,\pico\ampere$ at a frequency of $30\,\hertz$. The output
channel (i.e., the channel used to inject current into the presynaptic
cell) is number 0, while the input channels are numbers 0 and 1, with
0 being the channel connected to the presynaptic cell and 1 that of
the postsynaptic cell. The default is to record the presynaptic cell
in current clamp and the postsynaptic one in voltage clamp, but the
option \verb+--cclamp+ allows to instruct the protocol to record both
channels in current clamp mode.

\section{Utility programs}
\progname contains some additional programs that are not protocols,
but that can be used in various occasions during an electrophysiology
experiment. They are described in brief in the following, and we refer
the user to their help (i.e., \inlineCode{lcg help utility_name}) for
an in depth description of their options.
\begin{itemize}
\item \inlineCode{lcg zero} outputs zero on all channels of a DAQ
board. It is particularly useful at the beginning of an experiment,
when the output of the DAQ board is undefined.
\item \inlineCode{lcg output} outputs a given value on some channels
of the DAQ board. It can be useful when it is required to hold a cell
with a certain current or to a certain voltage, for example in voltage
clamp experiments.
\item \inlineCode{lcg annotate} allows the user to add comments into
an existing H5 data file. It accepts two options, \verb+-f+ for the
filename and \verb+-m+ for the comment message. If no options are
specified, the program prompts the user for the file name and for a
message.
\end{itemize}
