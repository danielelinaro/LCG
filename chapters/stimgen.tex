Of course, if \progname were restricted only to the protocols
described in Chapter~\ref{chap:protocols}, it would constitute a
cheap replacement of commercial programs such as Pulse, which offer a
vast array of additional functionalities. What makes \progname
extremely powerful and \emph{versatile} is the possibility to define
arbitrary \emph{stimulus} files to be used in voltage and current
clamp experiments, described in this chapter, and the possibility to
use \emph{configuration} files to describe arbitrary experiments,
(e.g., dynamic clamp or closed loop experiments) defined as the
interactions of elementary objects. Configuration files and such
objects, called \emph{entities} in the context pf \progname, are
described in detail in Chapters~\ref{chap:configuration} and
\ref{chap:entities}, respectively. This chapter describes how to use
\progname to inject arbitrary waveforms, described in stimulus files.

In principle, all traditional electrophysiology protocols (i.e., those
that do not require real-time control over the recorded quantities),
can be implemented in \progname by using just two commands,
\verb+lcg-stimgen+ and \verb+lcg-stimulus+: the former produces
stim-files\footnote{The syntax of stim-files is explained in great
  detail in the companion manual, which must be read to properly
  understand the following instructions.} according to the
specification of the user, and the
latter applies the synthesized waveforms to the appropriate output
channels while at the same time recording from an arbitrary number of
input channels. 

As a matter of fact, all the protocols described in
Chapter~\ref{chap:protocols}, with the exception of f-I
curve and frequency clamp protocols described in Sections~\ref{sec:fi}
and \ref{sec:fclamp}, respectively, can be implemented using only
\verb+lcg-stimgen+ and \verb+lcg-stimulus+. The reason for their
existence is simply the possibility to define appropriate default
values, which make the application of the protocols even simpler.

We will first briefly summarize the rationale behind the meta-language
used to generate stimulation files and then describe how to generate
them using \verb+lcg-stimgen+ and how to use \verb+lcg-stimulus+ to
stimulate and record using stimulation files.

\section{Stimulation files}
Stimulation files, or in short \textit{stim-files}, are simply text
files that contain a piecewise description of a particular
waveform. More in detail, each stim-file is composed of a certain
number of lines, with each line representing an elementary
sub-waveform: the full waveform is therefore given by the
concatenation of all sub-waveforms. Each elementary sub-waveform is
described by 12 coefficients:
\begin{table}[h!!]
\centering
\begin{tabular}{cccccccccccc}
$T$ & code & $p_1$ & $p_2$ & $p_3$ & $p_4$ & $p_5$ & fix-seed & seed &
subcode & op & exp
\end{tabular}
%\caption{Coefficients required for the definition of a sub-waveform.}
\label{tab:subwav}
\end{table}

The meaning of the coefficients is the following:
\begin{table}[h!!]
\begin{tabular}{r|l}
$T$ & duration of the sub-waveform \\
code & code that identifies the sub-waveform \\
$p_1, \ldots, p_5$ & parameters that specify the sub-waveform \\
fix-seed & whether or not the seed should be fixed (only for
stochastic sub-waveforms) \\
seed & the seed to use (only for stochastic sub-waveforms) \\
subcode & the code of another sub-waveform, if algebraic operations
are to be performed \\
op & the algebraic operation to perform \\
exp & the exponent to which the sub-waveform should be raised
\end{tabular}
\label{tab:params}
\end{table}

Available codes range from 1 to 12 and identify 12 different types of
elementary sub-waveforms, such as constants, sinusoids, square waves,
Ornstein-Uhlenbeck stochastic process realizations, and so on. The
full list of available waveforms is described in detailed in the
companion manual, which can be found at
\url{http://danielelinaro.github.io/dynclamp/stimgen_manual.pdf}.

If a noisy waveform is to be used, it is possible to set the seed used
in the generation of the pseudo-random numbers. It is also possible to
specify whether the seed should be used or not: in the latter case, a
random seed is generated using the device \verb+/dev/random+,
available on most UNIX systems.

As mentioned previously, sub-waveforms can be combined to yield more
complex stimulation waveforms: this is not limited only to the
concatenation of elementary sub-waveforms, but can be implemented also
by applying algebraic operations to sets of sub-waveforms. The way
this is achieved is described in the companion manual.

Finally, each elementary sub-waveform can be raised to a particular
exponent (for example 0.5 to obtain the square root of a waveform):
this is achieved by setting the appropriate value for the last
parameter of the sub-waveform.

This description of stimulus files is by no means exhaustive: we
strongly encourage readers to familiarize themselves with the
definition of stimulation waveforms by reading carefully through the
companion manual.

\section{Generating stimulation files}
As mentioned previously, stimulation files can be generated in
\progname using the command \verb+lcg-stimgen+: this provides a
convenient interface to the stim-file language, and saves the user the
hassle of having to write stim-files manually. Also, it provides a way
of interacting directly with \verb+lcg-stimulus+ by feeding the output
of \verb+lcg-stimgen+ to \verb+lcg-stimulus+ by means of UNIX pipelines,
as will be described in the following section.

The syntax of \verb+lcg-stimgen+ is the following:
\begin{verbatim}
lcg-stimgen -o|--output filename [-a|--append] 
          <waveform_1> [option <value>] p_1 p_2 ... p_5
          <waveform_2> [option <value>] p_1 p_2 ... p_5
          ...
          <waveform_N> [option <value>] p_1 p_2 ... p_5
\end{verbatim}
Since \verb+lcg-stimgen+ can deal with an arbitrary number of
waveforms, the options that specify the output file name (\verb+-o+ or
\verb+--output+ switches) and whether the output should be appended to
an existing file (\verb+-a+ or \verb+--append+ switches) should
precede the definition of any sub-waveform. If no output file is
specified, \verb+lcg-stimgen+ will output the resulting stim-file to
standard output: this is useful both for testing purposes and for
usage in conjuction with \verb+lcg-stimulus+ as will be described in
the following.

Next comes the specification of each single sub-waveform with its
parameters: the first parameter (i.e, \verb+waveform_1+,
\verb+waveform_2+ and so on) is a code that represent the type of
waveform. At present, the accepted codes are \verb+dc+, \verb+ou+,
\verb+sine+, \verb+square+, \verb+saw+, \verb+chirp+, \verb+ramp+,
\verb+poisson-reg+, \verb+poisson-exp+, \verb+poisson-bi+,
\verb+gaussian+, \verb+alpha+. For the meaning of each one, see the
companion manual. The optional arguments for each sub-waveform are the
following:
\begin{table}[h!!]
\begin{tabular}{r|l}
-d & duration of the sub-waveform in seconds \\
-s & seed of the sub-waveform, also sets \verb+fix-seed+ accordingly \\
-e & exponent of the sub-waveform \\
-p & sum this sub-waveform and the following \\
-m & subtract the following sub-waveform from this one \\
-M & multiply this sub-waveform and the following \\
-D & divide this sub-waveform by the following \\
-E & compute the composite waveform: this option is required for the last waveform in a series
\end{tabular}
\label{tab:optargs}
\end{table}
\newline
Every short option has a corresponding long one, detailed in the
online help of \verb+lcg-stimgen+, accessible by issuing at a prompt the
following command:
\begin{lstlisting}
lcg-stimgen -h
\end{lstlisting}
Additionally, all other options ---except for the duration--- have default
values, which are also detailed in the online help of the command.

The positional arguments \verb+p_1+ to \verb+p_5+ come after the
optional arguments. Each waveform requires a certain number of
parameters: for instance, the constant waveform requires just one
parameter, whereas a sinusoidal waveform requires 4. The exact number
of parameters required by each sub-waveform is accessible by issuing
the command
\begin{lstlisting}
lcg-stimgen help <code>
\end{lstlisting}
where code is one of the sub-waveform codes described before: for
example, the command
\begin{lstlisting}
lcg-stimgen help sine
\end{lstlisting}
produces the following output:
\begin{verbatim}
Waveform `sine' takes 4 parameters:
   1) amplitude
   2) frequency (Hz)
   3) phase
   4) offset
\end{verbatim}
Note that, according to the argument passing convention on UNIX
systems, if one of the positional arguments (i.e., one of the
parameters of a sub-waveform) is negative (e.g., a negative amplitude
in a constant sub-waveform), the \textbf{full} list of positional
arguments should be preceded by a double dash sign, as in
\begin{lstlisting}
lcg-stimgen -o negative.stim dc -d 1 -- -100
\end{lstlisting}

\subsection{Examples}
The following commands generate the stimulus files presented in the
companion manual as examples.

\textbf{Example 1}: a positive $5\,\second$-long step preceded and
followed by a $2.5\,\second$-long constant of 0.
\begin{lstlisting}
lcg-stimgen dc -d 2.5 0 dc -d 5 2 dc -d 2.5 0
\end{lstlisting}

\textbf{Example 2}: a negative $5\,\second$-long step preceded and
followed by a $2.5\,\second$-long constant of 0. Note the double minus
sign used to delimit the beginning of the positional arguments, when
at least one of them is negative.
\begin{lstlisting}
lcg-stimgen dc -d 2.5 0 dc -d 5 -- -2 dc -d 2.5 0
\end{lstlisting}

\textbf{Example 3}: 2 $200\,\milli\second$-long Ornsteing-Uhlenbeck
waveforms with different mean interleaved by constant waveforms.
\begin{lstlisting}
lcg-stimgen dc -d 0.1 0 ou -d 0.2 -- -2 0.5 1 dc -d 0.05 0 ou -d 0.2 2 0.5 1 dc -d 0.1 0
\end{lstlisting}

\textbf{Example 4}: 2 $200\,\milli\second$-long Ornsteing-Uhlenbeck
waveforms with different time constants interleaved by constant
waveforms.
\begin{lstlisting}
lcg-stimgen dc -d 0.1 0 ou -d 0.2 2 0.5 1 dc -d 0.05 0 ou -d 0.2 2 0.5 10 dc -d 0.1 0
\end{lstlisting}

\textbf{Example 5}: 3 $200\,\milli\second$-long Ornsteing-Uhlenbeck
waveforms with identical parameters. The first two also have the same
seed, whereas the third one has no seed specified.
\begin{lstlisting}
lcg-stimgen dc -d 0.1 0 ou -d 0.2 -s 21 2 0.5 100 dc -d 0.05 0 ou -d 0.2 -s 21 2 0.5 100 dc -d 0.05 0 ou -d 0.2 2 0.5 100 dc -d 0.1 0
\end{lstlisting}

\textbf{Example 6}: a $5\,\second$-long sinusoidal waveform.
\begin{lstlisting}
lcg-stimgen dc -d 2.5 0 sine -d 5 3 1 0 0 dc -d 2.5 0
\end{lstlisting}

\textbf{Example 7}: a $5\,\second$-long square waveform.
\begin{lstlisting}
lcg-stimgen dc -d 2.5 0 square -d 5 3 1 50 dc -d 2.5 0
\end{lstlisting}

\textbf{Example 8}: a $5\,\second$-long saw-tooth waveform.
\begin{lstlisting}
lcg-stimgen dc -d 2.5 0 saw -d 5 3 1 50 dc -d 2.5 0
\end{lstlisting}

\textbf{Example 9}: a $5\,\second$-long chirp waveform.
\begin{lstlisting}
lcg-stimgen dc -d 2.5 0 chirp -d 5 4 1 10 dc -d 2.5 0
\end{lstlisting}

\textbf{Example 10}: a $5\,\second$-long ramp waveform. Note that the
ramp waveform starts at the last value of the previous waveform (or
zero, if there is no sub-waveform preceding the ramp) and ends at the
value specified as a parameter, which therefore does \textbf{not}
necessarily represent the total span of the ramp.
\begin{lstlisting}
lcg-stimgen dc -d 2.5 -1 ramp -d 5 4 dc -d 2.5 4
\end{lstlisting}

\textbf{Example 11}: unipolar $1.5\,\milli\second$-long square pulses
of amplitude 4 delivered at a frequency of $10\,\hertz$.
\begin{lstlisting}
lcg-stimgen dc -d 0.5 0 poisson-reg -d 1 -- 4 -10 1.5 dc -d 0.5 0
\end{lstlisting}

\textbf{Example 12}: exponential pulses with a decay time constant of
$5\,\milli\second$ and amplitude 4 delivered at a frequency of
$10\,\hertz$.
\begin{lstlisting}
lcg-stimgen dc -d 0.5 0 poisson-exp -d 1 -- 4 -10 5 dc -d 0.5 0
\end{lstlisting}

\textbf{Example 13}: bipolar $5\,\milli\second$-long square pulses
of amplitude 4 delivered at a frequency of $10\,\hertz$.
\begin{lstlisting}
lcg-stimgen dc -d 0.5 0 poisson-bi -d 1 -- 4 -10 5 dc -d 0.5 0
\end{lstlisting}

\textbf{Example 14}: same as example 13, with the difference that here
the pulses are not delivered regularly, but rather they are generated
by an exponential distribution with mean rate equal to $10\,\hertz$.
\begin{lstlisting}
lcg-stimgen dc -d 0.5 0 poisson-bi -d 1 4 10 5 dc -d 0.5 0
\end{lstlisting}

\textbf{Example 15}: three ``bursts'' of unipolar square pulses whose
occurrence times are exponentially distributed. The first two bursts
have the same seed (therefore producing the same sequence of events),
while the third burst has no seed specified.
\begin{lstlisting}
lcg-stimgen dc -d 0.5 0 poisson-reg -d 1 -s 43 4 10 5 dc -d 0.5 0 poisson-reg -d 1 -s 43 4 10 5 dc -d 0.5 0 poisson-reg -d 1 4 10 5 dc -d 0.5 0
\end{lstlisting}

\textbf{Example 16}: alpha function waveform.
\begin{lstlisting}
lcg-stimgen dc -d 0.5 0 alpha -d 1 4 15 50 200 dc -d 0.5 0
\end{lstlisting}

\textbf{Example 17}: a rectified sinusoidal waveform. This can be
accomplished by setting the exponent parameter to -1, which
effectively takes the absolute value, point by point, of the original
elementary sub-waveform.
\begin{lstlisting}
lcg-stimgen dc -d 0.5 0 sine -d 5 -e -1 4 1 0 0 dc -d 0.5 1
\end{lstlisting}

\textbf{Example 18}: positive part of a sinusoidal waveform. This can
be accomplished by setting the exponent parameter to 0, which takes,
point by point, the positive part of the original elementary sub-waveform.
\begin{lstlisting}
lcg-stimgen dc -d 0.5 0 sine -d 5 -e 0 4 1 0 0 dc -d 0.5 1
\end{lstlisting}

\textbf{Example 19}: sum of a sinusoidal and a Ornstein-Uhlenbeck
waveform.
\begin{lstlisting}
lcg-stimgen dc -d 0.5 0 sine -d 5 -p 1 1 0 0 ou -E 0 0.2 5 dc -d 0.5 0
\end{lstlisting}

\textbf{Example 20}: product of a sinusoidal and a Ornstein-Uhlenbeck
waveform.
\begin{lstlisting}
lcg-stimgen dc -d 0.5 0 sine -d 5 -M 1 1 0 0 ou -E 0 0.2 5 dc -d 0.5 0
\end{lstlisting}

\section{Using stimulation files}
Stimulation files are the preferred way to represent arbitrary
waveforms in \progname. To ensure flexibility, stim-files do not
contain information about the unit of measure of the stimulus: on the
contrary, the same stimulus can be used to represent, for instance, a
current waveform (in which case the unit of measure will be
$\pico\ampere$) or a voltage waveform (in which case the unit of
measure will be $\milli\volt$). In other words, the \textit{meaning}
of a particular waveform will depend on the \textit{context} it is
used in: we will discuss this in more detail when talking about
configuration files in Chap.~\ref{chap:configuration}.

In this section we will see how to use \verb+lcg-stimulus+ to perform
conventional voltage and current clamp experiments, which typically
(but not necessarily) require the usage of stimulation
files. \verb+lcg-stimulus+ is a general purpose command that
reads from and writes to as many channels as the data acquisition card
on your system supports.

The syntax of \verb+lcg-stimgen+ is the following:
\begin{verbatim}
lcg-stimulus [option <value>]
\end{verbatim}
The command supports a large set of options: here we describe the most
commonly used ones, but we encourage the user to read carefully the
help page of \verb+lcg-stimulus+, which can be accessed by issuing the
command
\begin{lstlisting}
lcg-stimulus -h
\end{lstlisting}
The majority of short options have a corresponding long one, detailed
in the online help of \verb+lcg-stimulus+: here we will use either
long or short options.

Input and output channels are specified by using the \verb+-I+ and
\verb+-O+ options, respectively. Since input and output from multiple
channels are supported, \verb+lcg-stimulus+ understands four different
ways of specifying channels:
\begin{enumerate}
\item a single value, as in \verb+-I 1+, which means that the program
  should read only from channel 1. Note that data acquisition cards
  typically label channels starting from 0, so that channel 1 is
  effectively the \textit{second} input channel of the board.
\item a list of comma separated values, as in \verb+-I 0,1,2+, which
  instructs \verb+lcg-stimulus+ to read from the first three input
  channels.
\item a range of values, as in \verb+-I 0-2+, which is equivalent to
  \verb+-I 0,1,2+.
\item the string \verb+none+, as in \verb+-I none+, which instructs
  \verb+lcg-stimulus+ to use no input channels. The same is obviously
  valid also for output channels.
\end{enumerate}

Other options related to input and output are the following:
\begin{itemize}
\item \verb+-F+ to specify the sampling rate.
\item \verb+-D+ to specify the path of the device the data acquisition
  card is mapped to. In the case of comedi devices, this is typically
  \verb+/dev/comedi0+.
\item \verb+--input-subdevice+ and \verb+--output-subdevice+ to
  specify the input and output subdevices to use. In the case of
  National Instruments data acquisition cards, these are gerally equal
  to 0 and 1, respectively.
\item \verb+--input-gains+ to specify the input conversion factors. If
  only one value is specified, then the same conversion factor is used
  for all input channels. Alternatively, the user can specify as many
  conversion factors as there are input channels, separating them with
  commas, as in \verb+--input-gains 100,1,20+.
\item \verb+--output-gains+ to specify the output conversion factors.
\item \verb+--input-units+ and \verb+--output-units+ to specify the
  input and output units for each channel. As described previously for
  the input and output conversion factor, the user can either specify
  a single value (such as \verb+pA+ or \verb+mV+), or as many values
  as there are input and output channels, as in \verb+--output-units pA,V+.
\end{itemize}

Other important options are:
\begin{itemize}
\item \verb+-n+ to specify the number of trials.
\item \verb+-i+ to specify the interval (in seconds) between trials.
\item \verb+-o+ to specify the output file name. If this option is not
  passed, the program generates a file name based on the current date
  and time.
\end{itemize}

The user can specify the stimulation files to use with the \verb+-s+
option: in the case a single output channel is used, the simplest
scenario is that in which a single stimulation file is passed to the
program. In the case in which multiple stimulation files are passed to
\verb+lcg-stimulus+, the program will apply them sequentially to the
output channel. If multiple output channels are used and just one
stimulation file is specified, then the same output is fed to all
output channels. Alternatively, the user can specify as many
stimulation files as there are output channels, and
\verb+lcg-stimulus+ will apply each stimulation file to an output
channel, in the order in which they were passed at the command
line. The duration of the recording is equivalent to that of the
longest stimulus.

An alternative way to specify stimulation files is to use the
\verb+-d+ option: this tells \verb+lcg-stimulus+ to look for
stimulation files (i.e., all files that have a \verb+stim+ suffix) in
a specific directory. The same rules concerning number of output
channels and stimulation files outlined previously apply also in this
case.

As mentioned previously, \verb+lcg-stimulus+ can be used without
stimulation files: in this case, the program only records from the set
of input channels specified, without applying any stimulation. The
duration of the recording has to be specified with the \verb+-l+
option.

\subsection{Default values and pipelines}
Most \progname commands use environment variables to store default
values that rarely change, such as the path of the device that
represents your data acquisition cards: consequently, only a very
limited subset of \verb+lcg-stimulus+ options need to be passed on the
command line during the execution of routine protocols. For example,
in the simple but very common scenario in which the user wants to
record from one channel while injecting a stimulation current, the
following command can be used:
\begin{lstlisting}
lcg-stimulus -s current.stim -n 10 -i 5
\end{lstlisting}
where \verb+current.stim+ is a stim-file generated with
\verb+lcg-stimgen+ and the \verb+-n+ and \verb+-i+ options instruct
the program to run the stimulation 10 times with an interval of
$5\,\second$. \\
The full list of \progname-related environment variables can be found
in Sec.~\ref{sec:envvar}.

Further flexibility in the on-the-fly execution of simple protocols is
provided by pipelines\footnote{If you don't know what a UNIX pipeline
  is, read \href{http://en.wikipedia.org/wiki/Pipeline\_\%28Unix\%29}{this
    page} on Wikipedia.}: these allow feeding the output of
\verb+lcg-stimgen+ to \verb+lcg-stimulus+, therefore removing the need
of writing to file the stimulation file. The usage of pipelines is
best explained with an example: suppose that you want to inject a
$500\,\milli\second$-long step of depolarizing current of amplitude
$300\,\pico\ampere$ into a cell, preceded and followed by $2\,\second$
of baseline recording. The following one-line command performs such
stimulation:
\begin{lstlisting}
lcg-stimgen dc -d 2 0 dc -d 0.5 300 dc -d 2 0 | lcg-stimulus
\end{lstlisting}
Note however that pipelines can be used only when one output channel
is required. Another example that highlights the power and flexibility
of using \verb+lcg-stimgen+ and \verb+lcg-stimulus+ in conjunction is
the following:
\begin{lstlisting}
lcg-stimgen dc -d 1 0 poisson-reg -d 0.5 -- 2000 -20 1 dc -d 1 0 | lcg-stimulus -I 0,1 -O 0 -n 30 -i 5
\end{lstlisting}
This command can be used, for instance, to test the presence of a
connection between two cells recorded intracellularly simultaneously:
\verb+lcg-stimgen+ generates a $20\,\hertz$ train of
$1\,\milli\second$-long pulses with an amplitude of $2\,\nano\ampere$,
preceded and followed by a $1\,\second$-long baseline. The duration of
the train is $0.5\,\second$, which, at $20\,\hertz$, implies that 10
pulses will be delivered. \verb+lcg-stimulus+ outputs the stimulus on
channel 0 (connected to the putative presynaptic cell), while
simultaneously recording from channels 0 and 1. The stimulation is
performed 30 times at $5\,\second$ intervals. Notice that, to test the
other connection, it is sufficient to change the output channel to 1,
assuming that that is the output channel connected to the other cell
in the pair. This combination of \verb+lcg-stimgen+ and
\verb+lcg-stimulus+ is effectively equivalent to the \verb+lcg-pulses+
command, described in Sec.~\ref{sec:train}.

Note that, even though when piping the output of \verb+lcg-stimgen+ to
\verb+lcg-stimulus+ no stim-file is explicitly written to disk, every
time an \progname command is run all the metadata\sidenote{Metadata} necessary for
reconstructing unambiguously the performed protocol is stored in a
hidden directory in the folder where the command is run: such metadata
includes the stim-file and configuration file used. More information
about this can be found in Sec.~\ref{sec:metadata}.

\subsection{Examples}
As mentioned previously, the majority of the electrophysiology
protocols described in Chap.~\ref{chap:protocols} can be realized
simply by using \verb+lcg-stimgen+ in conjunction with
\verb+lcg-stimulus+. In this section, we present a series of
commands or short scripts that reproduce some of the electrophysiology
protocols previously described.

\subsubsection{Depolarizing or hyperpolarizing pulse protocol}
In this case, the same stimulation is repeated 20 times, so it is
appropriate to pipe the output of \verb+lcg-stimgen+ to \verb+lcg-stimulus+
\begin{lstlisting}
lcg-stimgen dc -d 0.5 0 dc -d 0.001 2000 dc -d 0.5 0 | lcg-stimulus -n 20 -i 5
\end{lstlisting}
To inject a hyperpolarizing pulse of current, it is sufficient to
change the parameters of the second constant waveform in the call to
\verb+lcg-stimgen+.

\subsubsection{Depolarizing or hyperpolarizing steps protocol}
For this protocol, we first create a directory and store all the
stimulation files in it and then run \verb+lcg-stimulus+ specifying the
directory containing the stim-files.
\begin{lstlisting}
mkdir stimuli
for amp in `seq -300 50 50` ; do
    lcg-stimgen -o stimuli/step_$amp.stim dc -d 0.5 0 dc -d 1 $amp dc -d 0.5 0
done
lcg-stimulus -d stimuli -n 2 -i 1
\end{lstlisting}

\subsubsection{Ramp protocol}
For this protocol, we want the ramp to start from $50\,\pico\ampere$
and end at $300\,\pico\ampere$. This can be accomplished by summing a
constant with amplitude $50\,\pico\ampere$ and a ramp with final value
$250\,\pico\ampere$.
\begin{lstlisting}
lcg-stimgen dc -d 0.5 0 dc -p -d 30 50 ramp -E 300 dc -d 0.5 0 | lcg-stimulus -n 2 -i 30
\end{lstlisting}

\subsubsection{Input resistance protocol}
This protocol is similar to the train of pulses described previously
and in Sec.~\ref{sec:train}, and can be accomplished with the
following command:
\begin{lstlisting}
lcg-stimgen dc -d 1 0 poisson-reg -d 90 -- -100 1 300 | lcg-stimulus
\end{lstlisting}

\subsubsection{Recording of spontaneous activity}
To record spontaneous activity, a stimulus file is not necessary, but
we must specify the duration of the recording, using the \verb+-l+
option of \verb+lcg-stimulus+.
\begin{lstlisting}
lcg-stimulus -l 60 -I 0-3
\end{lstlisting}