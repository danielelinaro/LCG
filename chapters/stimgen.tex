Of course, if \progname were restricted only to the protocols
described in Chapter~\ref{chap:protocols}, it would constitute a
cheap replacement of commercial programs such as Pulse, which offer a
vast array of additional functionalities. What makes \progname
extremely powerful and \emph{versatile} is the possibility to define
arbitrary \emph{stimulus} files to be used in voltage and current
clamp experiments, described in this chapter, and the possibility to
use \emph{configuration} files to describe arbitrary experiments,
(e.g., dynamic clamp or closed loop experiments) defined as the
interactions of elementary objects. Configuration files and such
objects, called \emph{entities} in the context pf \progname, are
described in detail in Chapters~\ref{chap:configuration} and
\ref{chap:entities}, respectively. This chapter describes how to use
\progname to inject arbitrary waveforms, described in stimulus files.

In principle, all traditional electrophysiology protocols (i.e., those
that do not require real-time control over the recorded quantities),
can be implemented in \progname by using just two commands,
\verb+lcg-stimgen+ and \verb+lcg-stimulus+: the former produces
stim-files\footnote{The syntax of stim-files is explained in great
  detail in the companion manual, which must be read to properly
  understand the following instructions.} according to the
specification of the user, and the
latter applies the synthesized waveforms to the appropriate output
channels while at the same time recording from an arbitrary number of
input channels. 

As a matter of fact, all the protocols described in
Chapter~\ref{chap:protocols}, with the exception of f-I
curve and frequency clamp protocols described in Sections~\ref{sec:fi}
and \ref{sec:fclamp}, respectively, can be implemented using only
\verb+lcg-stimgen+ and \verb+lcg-stimulus+. The reason for their
existence is simply the possibility to define appropriate default
values, which make the application of the protocols even simpler.

