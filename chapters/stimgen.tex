Of course, if \progname were restricted only to the protocols
described in Chapter~\ref{chap:protocols}, it would constitute a
cheap replacement of commercial programs such as Pulse, which offer a
vast array of additional functionalities. What makes \progname
extremely powerful and \emph{versatile} is the possibility to define
arbitrary \emph{stimulus} files to be used in voltage and current
clamp experiments, described in this chapter, and the possibility to
use \emph{configuration} files to describe arbitrary experiments,
(e.g., dynamic clamp or closed loop experiments) defined as the
interactions of elementary objects. Configuration files and such
objects, called \emph{entities} in the context pf \progname, are
described in detail in Chapters~\ref{chap:configuration} and
\ref{chap:entities}, respectively.

This chapter describes how to use \progname to inject arbitrary
waveforms, described in stimulus files.
